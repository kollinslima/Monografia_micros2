\chapter{Conclus�o}
\label{Conclusao}

\par Neste trabalho foi apresentado uma proposta para realizar o reconhecimento de sinais, aplicada � L�ngua Brasileira de Sinais - LIBRAS.
\par Pode-se dizer que o trabalho atinge parcialmente seus objetivos. Por um lado, foi implementado um sistema capaz de ler, processar, classificar e responder textualmente os s�mbolos apresentados por meio da c�mera de um \textit{smartphone} Android. Por outro, com uma taxa de erro superior � 60\%, o sistema n�o � capaz de fazer a classifica��o correta para todos os s�mbolos num�ricos propostos.

\par Apesar dos resultados obtidos, o desenvolvimento deste projeto exigiu conhecimento de diversas �reas, o que foi de grande valia para o aprendizado. Entre os conhecimentos obtidos, pode-se destacar as diversas t�cnicas de processamento de imagem que foram necess�rias na etapa de pr�-processamento (com aux�lio da biblioteca \textit{OpenCV}); o uso da Raspberry-pi como elemento central no projeto; o uso do \textit{Flask} para desenvolver a interface \textit{web}; o uso do banco de dados \textit{MongoDB} al�m da pr�tica em desenvolvimento Python e Android.

\par Por fim, analisando o que foi desenvolvido, notou-se que esta proposta n�o � suficiente para prover independ�ncia ao indiv�duo surdo, que o objetivo principal. Isso se deve ao fato de o sistema n�o ser de f�cil uso no dia-a-dia, mesmo se estivesse em perfeito funcionamento, uma vez que depende de uma c�mera apontando para o usu�rio, fazendo com que o aplicativo dependesse de duas pessoas para seu uso (uma segurando a c�mera e outra fazendo os gestos), do contr�rio restringiria os movimentos por exigir que uma das m�os segure a c�mera. Seu uso seria mais eficiente em um contexto de palestra ou algum outro evento onde j� existe uma c�mera externa monitorando o ambiente.

\par Para trabalhos futuros, � preciso pensar em outras maneiras de fazer o reconhecimento de gestos, que seja de f�cil uso por uma �nica pessoa e que n�o envolva uma quantidade excessiva de hardware externo, tanto para prover um sistema mais acess�vel, quanto devido � est�tica.


%\section*{Trabalhos futuros}
%
%Isso � para a Monografia Final de defesa.....











